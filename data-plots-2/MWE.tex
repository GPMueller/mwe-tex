\documentclass{report}
\usepackage[T1]{fontenc}
\usepackage[utf8]{inputenc}
\usepackage[english]{babel}

\usepackage[dvipsnames]{xcolor}

\usepackage{tikz,pgfplots,pgfplotstable}
\pgfplotsset{compat=newest}

%===================================================
\usetikzlibrary{positioning,plotmarks,external}
\tikzexternalize[mode=convert with system call,shell escape=-enable-write18]
%===================================================

%==== Externalisation ==============================
\makeatletter
\newcommand{\useexternalfile}[2][Tikz] % \useexternalfile[Folder]{Filename}
{
	\tikzsetnextfilename{#2-output}
	\input{#1/#2.tikz}
}
\makeatother
%===================================================

%==== Line Colors CMYK ==============================
%\definecolor{color1}{cmyk}{0.784,0.506,0,0}      % Blue
\colorlet{color1}{RoyalBlue}
\definecolor{color2}{cmyk}{0,0.886,0.877,0.106}  % Red
\definecolor{color3}{cmyk}{0,0.502,1,0}          % Orange
\definecolor{color4}{cmyk}{0.56,0,0.577,0.314}   % Green
%\definecolor{color5}{cmyk}{0.0675,0.521,0,0.361} % Purple
\definecolor{color5}{cmyk}{0.62,0.79,0,0} % Purple
\definecolor{color6}{cmyk}{0,0,0.8,0}            % Yellow
\definecolor{color01}{cmyk}{0,0,0,0}             % White
\definecolor{color02}{cmyk}{0,0,0,1}             % Black
\definecolor{color03}{cmyk}{0,0,0,0.55}          % Gray
%\definecolor{pureblue}{cmyk}{1,1,0,0}
%\definecolor{puregreen}{cmyk}{1,0,1,0}
%\definecolor{purered}{cmyk}{0,1,1,0}
%===================================================
	
	
	
%==== Plot Styles  =================================
\pgfplotsset
{
    lpPlotStyle/.style=
	{
		line width=0.8,
		mark=*,
		mark size=0.6
	},
	lPlotStyle/.style=
	{
		line width=0.8,
		mark size=0.0
	},
	pPlotStyle/.style=
	{
		only marks,
		mark=*,
		mark size=0.6
	},
    axisStyleA/.style=
    {
        axis line style =
        {
            very thick,
            shorten <=-0.5\pgflinewidth,
            Black!55
        },
        ylabel near ticks,
        label style=
        {
            font=\small%\inplottext
        },
%			unit marking pre={\inplottextsize\lbrack}, %\inplottext \mathrm{in}},
%			unit marking post={\inplottextsize\rbrack},
        ticklabel style=
        {
            color=Black!55,
            %font=\ssfamily % % Think about sans serif font for ticklabels!
            font=\small%\inplottextsize % don't use non-math font for numbers
        },
        tick style=
        {
            thick
        },
        axis lines=left,
        y tick scale label style={xshift=-1em}
    }
}
%===================================================
	
%==== fix the axis being below the plot lines ======
%	\pgfplotsset
%	{
%		axis line on top/.style=
%		{
%			axis line style=transparent,
%			ticklabel style=transparent,
%			tick style=transparent,
%			axis on top=false,
%			after end axis/.append code=
%			{
%				\pgfplotsset
%				{
%					axis line style=opaque,
%					ticklabel style=opaque,
%					tick style=opaque,
%					grid=none,
%					every extra x tick/.style={grid=none},
%					every extra y tick/.style={grid=none}},
%					\pgfplotsdrawaxis
%				}
%			}
%		}
%	}
%===================================
	
	
	
	
\begin{document}
	
		
    \begin{tikzpicture}
    [
    %scale=0.7
    ]
    %==== Plot Sizes
    \newcommand{\plotwidth}{\linewidth*0.7}
    \newcommand{\plotheight}{3.0/5.0*\plotwidth}

    % % % ONE
    %==== Data: Tables
    \pgfplotstableread{data/example_1_1.txt}\datatableOne
    %==== Rx last value of datatable
    \pgfplotstablegetrowsof{data/example_1_1.txt}
    \pgfmathsetmacro{\nLinesOne}{\pgfplotsretval-1}
    \pgfplotstablegetelem{\nLinesOne}{Rx}\of{\datatableOne}
    \pgfmathsetmacro{\RxtotOne}{\pgfplotsretval}

    % % % TWO
    %==== Data: Tables
    \pgfplotstableread{data/example_1_2.txt}\datatableTwo
    %==== Rx last value of datatable
    \pgfplotstablegetrowsof{data/example_1_2.txt}
    \pgfmathsetmacro{\nLinesTwo}{\pgfplotsretval-1}
    \pgfplotstablegetelem{\nLinesTwo}{Rx}\of{\datatableTwo}
    \pgfmathsetmacro{\RxtotTwo}{\pgfplotsretval}

    % % % THREE
    %==== Data: Tables
    \pgfplotstableread{data/example_1_3.txt}\datatableThree
    %==== Rx last value of datatable
    \pgfplotstablegetrowsof{data/example_1_3.txt}
    \pgfmathsetmacro{\nLinesThree}{\pgfplotsretval-1}
    \pgfplotstablegetelem{\nLinesThree}{Rx}\of{\datatableThree}
    \pgfmathsetmacro{\RxtotThree}{\pgfplotsretval}


    % % % FOUR
    %==== Data: Tables
    \pgfplotstableread{data/example_1_4.txt}\datatableFour
    %==== Rx last value of datatable
    \pgfplotstablegetrowsof{data/example_1_4.txt}
    \pgfmathsetmacro{\nLinesFour}{\pgfplotsretval-1}
    \pgfplotstablegetelem{\nLinesFour}{Rx}\of{\datatableFour}
    \pgfmathsetmacro{\RxtotFour}{\pgfplotsretval}


    \begin{axis}
    [
        axisStyleA,
        %		legendStyleInside,
        xlabel=$x$,
        xlabel style={name=xlabel},
        % x unit={\siunit{s}},
        ylabel=$y$,
        % y unit={\siunit{m}},
        %		ymax=-22.74,
        scale only axis,
        width=\plotwidth,
        height=\plotheight,
        %		at={(\plotwidth*1.0/5.0,0)},
        %		legend entries={one,other},
        legend pos= north west
    ]
    % % % ONE
    \addplot
    [
        lpPlotStyle,
        color=color1,
        %			smooth,
        %			mark=none
    ] table [x expr=\thisrow{Rx}/\RxtotOne, y expr=\thisrow{Ry}] {\datatableOne};
    \addlegendentry{\small Test 1};

    % % % TWO
    \addplot
    [
        lpPlotStyle,
        color=color2,
    ] table [x expr=\thisrow{Rx}/\RxtotTwo, y expr=\thisrow{Ry}] {\datatableTwo};
    \addlegendentry{\small Test 2};


    % % % THREE
    \addplot
    [
        lpPlotStyle,
        color=color3,
    ] table [x expr=\thisrow{Rx}/\RxtotThree, y expr=\thisrow{Ry}] {\datatableThree};
    \addlegendentry{\small Test 3};


    % % % FOUR
    \addplot
    [
        lpPlotStyle,
        color=color4,
    ] table [x expr=\thisrow{Rx}/\RxtotFour, y expr=\thisrow{Ry}] {\datatableFour};
    \addlegendentry{\small Test 4};

    \end{axis}



    \end{tikzpicture}
	
\end{document}