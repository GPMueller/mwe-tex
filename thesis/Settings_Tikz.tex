
%==== General Stuff ================================
\usepackage{tikz,pgfplots,pgfplotstable,pgffor,pgfmath,tikz-3dplot}
%\usepgfplotslibrary{external}
%\usepackage{underscore} % to read in tables with underscores in head
%\usepackage{etoolbox} % for patching underscores in tikz tables

%\usepackage[margin=15mm]{geometry}

\usetikzlibrary{positioning,plotmarks,pgfplots.colormaps,external,3d,calc,arrows}
\usetikzlibrary{decorations,decorations.pathmorphing,decorations.pathreplacing}
\usetikzlibrary{shapes,arrows}
\tikzexternalize[prefix=output/,mode=convert with system call,shell escape=-enable-write18]
%\tikzsetexternalprefix{\maintex/PBC/output/}

\usepgfplotslibrary{units}
\usepgfplotslibrary{groupplots}
%===================================================


%==== PGF table column names key ============================
% see also http://tex.stackexchange.com/a/156428/70901

% taken from one comment of Villemoes in
% http://tex.stackexchange.com/a/15263/3061
\def\pgfglobalkeys#1{\begingroup \ifnum\the\globaldefs>0\relax \else \globaldefs=1\fi \pgfkeys{#1}\endgroup}

% define the new key
\pgfplotstableset{display column names/.code={
        \foreach[count=\n from 0] \x in {#1} {
            \pgfglobalkeys{/pgfplots/table/display columns/\n/.estyle={column name=\x}}
        }
    }
}
%===================================================


%==== Tikz & PGF Camera alignment macro ========================
% Style to set TikZ camera angle, like PGFPlots `view`
\tikzset{viewport/.style 2 args={
    x={({cos(-#1)*1cm},{sin(-#1)*sin(#2)*1cm})},
    y={({-sin(-#1)*1cm},{cos(-#1)*sin(#2)*1cm})},
    z={(0,{cos(#2)*1cm})}
}}
%===================================================

%==== Arrows ===================================
\tikzset
{
	%Define standard arrow tip
	>=stealth',
}
%    %Define style for boxes
%    punkt/.style={
%           rectangle,
%           rounded corners,
%           draw=black, very thick,
%           text width=6.5em,
%           minimum height=2em,
%           text centered},
%    % Define arrow style
%    pil/.style={
%           ->,
%           thick,
%           shorten <=2pt,
%           shorten >=2pt,}
%}
%============================================


%==== Externalisation ==============================
\makeatletter
\newcommand{\useexternalfile}[2][Tikz] % \useexternalfile[Folder]{Filename}
{
	\tikzsetnextfilename{#2-output}
%	\begingroup
%		\tikzset{every picture/.append style={scale=#1}}%
%		\input{\tikzexternal@filenameprefix#2.tikz}	
%	\endgroup
	\input{#1/#2.tikz}
}
%\newcommand{\useexternalfile}[1]
%{
%	\tikzsetnextfilename{#1-output}
%	%\input{\tikzexternal@filenameprefix#1.tikz}
%	\input{#1.tikz}
%}
\makeatother
%===================================================




%==== Nodes Near Some Coords =======================
\pgfplotsset{
    node near coord/.style args={#1/#2/#3}{% Style for activating the label for a single coordinate
        nodes near coords*={
            \ifnum\coordindex=#1 #2\fi
        },
        scatter/@pre marker code/.append code={
            \ifnum\coordindex=#1 \pgfplotsset{every node near coord/.append style=#3}\fi
        }
    },
    nodes near some coords/.style={ % Style for activating the label for a list of coordinates
        scatter/@pre marker code/.code={},% Reset the default scatter style, so we don't get coloured markers
        scatter/@post marker code/.code={},% 
        node near coord/.list={#1} % Run "node near coord" once for every element in the list
    }
}
%==== Pins Near Some Coords ========================
\pgfplotsset{
    pin near coord/.style args={#1/#2}{% Style for activating the label for a single coordinate
        scatter/@pre marker code/.append code={
            \ifnum\coordindex=#1 \node[pin=#2]{};\fi
        }
    },
    pins near some coords/.style={ % Style for activating the label for a list of coordinates
        scatter,
        scatter/@pre marker code/.code={},% Reset the default scatter style, so we don't get coloured markers
        scatter/@post marker code/.code={},% 
        pin near coord/.list={#1} % Run "pin near coord" once for every element in the list
    }
}
%===================================================



%==== Select lines or columns from Datafile ========
\pgfplotsset
{
	select coords between index/.style 2 args=
	{
		x filter/.code=
		{
			\ifnum\coordindex<#1\def\pgfmathresult{}\fi
			\ifnum\coordindex>#2\def\pgfmathresult{}\fi
		}
	}
}

\pgfplotsset{
	discard if not/.style 2 args={
		x filter/.code={
			\edef\tempa{\thisrow{#1}}
			\edef\tempb{#2}
			\ifx\tempa\tempb
			\else
			\def\pgfmathresult{inf}
			\fi
		}
	}
}
%===================================================


%==== Harpoons for vectors
% See   http://tex.stackexchange.com/questions/66723/nudging-overset-characters-downwards
%  and http://tex.stackexchange.com/questions/23411/tikzpicture-in-node-of-another-tikzpicture-how-to-screen-of-from-inheriting-sty
% Just for fun, a tikz solution
\renewcommand{\vec}[1]{
	\tikzset{external/export next=false}
	\tikz[baseline=(X.base),inner sep=0pt,outer sep=0pt] {
		\node[inner sep=0pt,outer sep=0pt] (X) {$#1$};
		\node[anchor=center,yshift=1.5pt,xshift=0.4pt] at (X.north) {$\scriptscriptstyle\rightharpoonup$};
	}
}
%===================



%%==== Fix Underscores in table read ================
%\makeatletter
%% Patch the command that precedes reading the table
%\patchcmd{\pgfplotstablereadpreparecatcodes@}
%  {\pgfplotstableinstallignorechars}
%  {\catcode`\_=13 \begingroup\lccode`~=`_ 
%   \lowercase{\endgroup\let~}\textunderscore
%   \pgfplotstableinstallignorechars}
%  {}{}
%% Patch the command that restores the category codes
%\patchcmd{\pgfplotstablereadpreparecatcodes}
%  {\noexpand}
%  {\catcode\string`\noexpand\_=\the\catcode\string`\_ \noexpand}
%  {}{}
%\makeatother
%
%%% For your convenience: This is the unaltered macro that prints
%%% the tick label when "xticklabels from table" is used
%\makeatletter
%\def\pgfplots@user@ticklabel@list@x{%
%    \pgfplotslistselectorempty\ticknum\of\pgfplots@xticklabels\to\tick
%    \tick
%}
%\makeatother
%%===================================================
